\documentclass[a4paper, 11pt]{article}
\usepackage{comment} % enables the use of multi-line comments (\ifx \fi) 
\usepackage{lipsum} %This package just generates Lorem Ipsum filler text. 
\usepackage{fullpage} % changes the margin
\usepackage{listings}
\usepackage{graphicx}
\usepackage{amsmath}
\usepackage{subcaption}
\usepackage[margin=1in]{geometry}
\usepackage{pdfpages}

\begin{document}

\newpage

%Header-Make sure you update this information!!!!
\noindent
\large\textbf{MPhil project proposal} \hfill \textbf{Marton Havasi} \\
\normalsize Supervisor: Jos\'e Miguel Hern\'andez-Lobato \hfill Due Date: 17/03/16

\section*{Designing neural network hardware accelerators with deep Gaussian processes}

\subsection*{Project description}

Bayesian optimization methods are often employed in a hardware design setting where simulation of a given hardware configuration can be computationally very expensive. The problem gets more difficult when there are multiple objective functions to balance against each other, for instance, the power consumption and the accuracy of the hardware accelerator. The set of optimal configurations that are not outperformed in both objective functions is called the Pareto front. This project attempts to use deep Gaussian process models in order to give better prediction of the Pareto front of multiobjective optimization problems. The state-of-the-art algorithms use Gaussian processes to model each objective function. This approach can be inefficient because it is unable to capture the correlation between the different objectives. Deep Gaussian processes can improve on this by sharing multiple layers of Gaussian processes between the objective functions. By capturing these dependencies, we can increase the accuracy of the model and therefore increase the accuracy of the resulting Pareto front.

The project will have three major parts:
\begin{itemize}
\item
Implementing deep Gaussian processes for multiple outputs. We have code available for deep Gaussian processes, however, this code needs to be extended to be able to cope with multiple outputs that share the hidden layers. This will be used to model the objective function. \cite{dgps}
\item
Using the previously implemented model in Bayesian optimization. For this, we need to use a multiobjective acquisition function. Our current candidate is SMSego \cite{smsego} which is a technique that was successfully employed with Gaussian processes. We aim to extend it with our deep Gaussian model.

At this point, we can measure the performance of our model on toy problems. For example, we can generate test data by sampling objective functions from the deep Gaussian process model. We can use the traditional approach as the baseline for the experiments.
\item
Putting the deep Gaussian process model to the test in a real-world optimization problem. We will be using it to design hardware accelerators for neural networks. The simulation of hardware configurations is very expensive and therefore they are  ideal candidates for optimization methods. We aim to accurately predict the Pareto front of the configurations. The performance of deep Gaussian processes can be compared to the performance of the already existing approaches. \cite{hardware} \cite{acc}
\end{itemize}

\subsection*{Success criteria}

\begin{itemize}
\item
Deep Gaussian processes for multiple objectives show improvement over the approach where the objectives are modeled independently on the previously described toy problems.
\item
The performance of deep Gaussian processes in the hardware design problem is on-par with the currently existing approaches.
\end{itemize}

\subsection*{Extensions}

\begin{itemize}
\item
Extend the design of neural network hardware accelerators to larger datasets. Currently, most of the work is being done on the MNIST dataset because of its small size. However, there have been recent developments in the simulation software so it is within reach to test on larger datasets. We want to consider hardware accelerators for the ImageNet dataset.
\item
Printing the hardware design. If we manage to find a high-performing configuration, it can be realized for further testing.
\end{itemize}

\subsection*{Workplan}

\begin{itemize}
\item March 20. - April 30.

\textbf{Focus:} Coursework and exams. Preliminary reading.

\item May 1. - May 14.

\textbf{Focus:} Literature review on deep Gaussian processes.

\textbf{Deliverable:} Implementation for multiobjective deep Gaussian processes.

\item May 15. - May 28.

\textbf{Focus:} Literature review on multiobjective acquisition functions. Extending the 
implementation of SMSego with deep Gaussian models.

\textbf{Deliverable:} Implementation of multiobjective Bayesian optimization methods with deep Gaussian models.

\item May 29. - June 11.

\textbf{Focus:} Evaluation of deep Gaussian models on toy problems.

\textbf{Deliverable:} Report on the performance on toy problems.

\item June 12. - June 25.

\textbf{Focus:} Applying deep Gaussian models to the hardware design problem.

\textbf{Deliverable:} Implementation of the deep Gaussian models to the hardware design problem.

\item June 26. - July 9.

\textbf{Focus:} Evaluating the performance of deep Gaussian models on the hardware design problem.

\textbf{Deliverable:} Report on the performance of the deep Gaussian models on the hardware design problem.

\item July 10. - July 23.

\textbf{Focus:} This period is left for extensions if the project is progressing well. Otherwise this period will be used to catch up to schedule.

\item July 24. - August 11.

\textbf{Focus:} Finishing the write-up.

\textbf{Deliverable:} Thesis.


\end{itemize}



\begin{thebibliography}{9}
\bibitem{dgps} 
Bui T. D., Hern\'andez-Lobato J. M., Li Y., Hern\'andez-Lobato D. and Turner R. E. Deep
\textit{Gaussian Processes for Regression using Approximate Expectation Propagation}, In ICML,
2016.
 
\bibitem{smsego} 
W. Ponweiser, T. Wagner, D. Biermann, and M. Vincze,  
\textit{Multiobjective optimization on a limited budget of evaluations using model-assisted s-metric selection} in Proc. Parallel Problem Solving from Nature (PPSN X) , Dortmund, Germany, Sept. 2008, pp. 784–794. 

\bibitem{hardware} 
Hern\'andez-Lobato J. M., Gelbart M. A., Reagen B., Adolf R., Hern\'andez-Lobato D., What-
mough P., Brooks D., Wei G.-Y. and Adams R. P. \textit{Designing Neural Network Hardware
Accelerators with Decoupled Objective Evaluations}, In NIPS Workshop on Bayesian Opti-
mization, Barcelona, Spain, 2016.

\bibitem{acc}
Reagen B., Whatmough P. Adolf R., Rama S., Lee H., Lee S., Hern\'andez-Lobato J. M., Wei
G. Y. and Brooks D. \textit{Minerva: Enabling Low-Power, High-Accuracy Deep Neural Network
Accelerators}, In ISCA, 2016.
\end{thebibliography}



\end{document}
